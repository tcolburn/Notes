
% Default to the notebook output style

    


% Inherit from the specified cell style.




    
\documentclass{article}

    
    
    \usepackage{graphicx} % Used to insert images
    \usepackage{adjustbox} % Used to constrain images to a maximum size 
    \usepackage{color} % Allow colors to be defined
    \usepackage{enumerate} % Needed for markdown enumerations to work
    \usepackage{geometry} % Used to adjust the document margins
    \usepackage{amsmath} % Equations
    \usepackage{amssymb} % Equations
    \usepackage{eurosym} % defines \euro
    \usepackage[mathletters]{ucs} % Extended unicode (utf-8) support
    \usepackage[utf8x]{inputenc} % Allow utf-8 characters in the tex document
    \usepackage{fancyvrb} % verbatim replacement that allows latex
    \usepackage{grffile} % extends the file name processing of package graphics 
                         % to support a larger range 
    % The hyperref package gives us a pdf with properly built
    % internal navigation ('pdf bookmarks' for the table of contents,
    % internal cross-reference links, web links for URLs, etc.)
    \usepackage{hyperref}
    \usepackage{longtable} % longtable support required by pandoc >1.10
    \usepackage{booktabs}  % table support for pandoc > 1.12.2
    \usepackage{ulem} % ulem is needed to support strikethroughs (\sout)
    

    
    
    \definecolor{orange}{cmyk}{0,0.4,0.8,0.2}
    \definecolor{darkorange}{rgb}{.71,0.21,0.01}
    \definecolor{darkgreen}{rgb}{.12,.54,.11}
    \definecolor{myteal}{rgb}{.26, .44, .56}
    \definecolor{gray}{gray}{0.45}
    \definecolor{lightgray}{gray}{.95}
    \definecolor{mediumgray}{gray}{.8}
    \definecolor{inputbackground}{rgb}{.95, .95, .85}
    \definecolor{outputbackground}{rgb}{.95, .95, .95}
    \definecolor{traceback}{rgb}{1, .95, .95}
    % ansi colors
    \definecolor{red}{rgb}{.6,0,0}
    \definecolor{green}{rgb}{0,.65,0}
    \definecolor{brown}{rgb}{0.6,0.6,0}
    \definecolor{blue}{rgb}{0,.145,.698}
    \definecolor{purple}{rgb}{.698,.145,.698}
    \definecolor{cyan}{rgb}{0,.698,.698}
    \definecolor{lightgray}{gray}{0.5}
    
    % bright ansi colors
    \definecolor{darkgray}{gray}{0.25}
    \definecolor{lightred}{rgb}{1.0,0.39,0.28}
    \definecolor{lightgreen}{rgb}{0.48,0.99,0.0}
    \definecolor{lightblue}{rgb}{0.53,0.81,0.92}
    \definecolor{lightpurple}{rgb}{0.87,0.63,0.87}
    \definecolor{lightcyan}{rgb}{0.5,1.0,0.83}
    
    % commands and environments needed by pandoc snippets
    % extracted from the output of `pandoc -s`
    \providecommand{\tightlist}{%
      \setlength{\itemsep}{0pt}\setlength{\parskip}{0pt}}
    \DefineVerbatimEnvironment{Highlighting}{Verbatim}{commandchars=\\\{\}}
    % Add ',fontsize=\small' for more characters per line
    \newenvironment{Shaded}{}{}
    \newcommand{\KeywordTok}[1]{\textcolor[rgb]{0.00,0.44,0.13}{\textbf{{#1}}}}
    \newcommand{\DataTypeTok}[1]{\textcolor[rgb]{0.56,0.13,0.00}{{#1}}}
    \newcommand{\DecValTok}[1]{\textcolor[rgb]{0.25,0.63,0.44}{{#1}}}
    \newcommand{\BaseNTok}[1]{\textcolor[rgb]{0.25,0.63,0.44}{{#1}}}
    \newcommand{\FloatTok}[1]{\textcolor[rgb]{0.25,0.63,0.44}{{#1}}}
    \newcommand{\CharTok}[1]{\textcolor[rgb]{0.25,0.44,0.63}{{#1}}}
    \newcommand{\StringTok}[1]{\textcolor[rgb]{0.25,0.44,0.63}{{#1}}}
    \newcommand{\CommentTok}[1]{\textcolor[rgb]{0.38,0.63,0.69}{\textit{{#1}}}}
    \newcommand{\OtherTok}[1]{\textcolor[rgb]{0.00,0.44,0.13}{{#1}}}
    \newcommand{\AlertTok}[1]{\textcolor[rgb]{1.00,0.00,0.00}{\textbf{{#1}}}}
    \newcommand{\FunctionTok}[1]{\textcolor[rgb]{0.02,0.16,0.49}{{#1}}}
    \newcommand{\RegionMarkerTok}[1]{{#1}}
    \newcommand{\ErrorTok}[1]{\textcolor[rgb]{1.00,0.00,0.00}{\textbf{{#1}}}}
    \newcommand{\NormalTok}[1]{{#1}}
    
    % Additional commands for more recent versions of Pandoc
    \newcommand{\ConstantTok}[1]{\textcolor[rgb]{0.53,0.00,0.00}{{#1}}}
    \newcommand{\SpecialCharTok}[1]{\textcolor[rgb]{0.25,0.44,0.63}{{#1}}}
    \newcommand{\VerbatimStringTok}[1]{\textcolor[rgb]{0.25,0.44,0.63}{{#1}}}
    \newcommand{\SpecialStringTok}[1]{\textcolor[rgb]{0.73,0.40,0.53}{{#1}}}
    \newcommand{\ImportTok}[1]{{#1}}
    \newcommand{\DocumentationTok}[1]{\textcolor[rgb]{0.73,0.13,0.13}{\textit{{#1}}}}
    \newcommand{\AnnotationTok}[1]{\textcolor[rgb]{0.38,0.63,0.69}{\textbf{\textit{{#1}}}}}
    \newcommand{\CommentVarTok}[1]{\textcolor[rgb]{0.38,0.63,0.69}{\textbf{\textit{{#1}}}}}
    \newcommand{\VariableTok}[1]{\textcolor[rgb]{0.10,0.09,0.49}{{#1}}}
    \newcommand{\ControlFlowTok}[1]{\textcolor[rgb]{0.00,0.44,0.13}{\textbf{{#1}}}}
    \newcommand{\OperatorTok}[1]{\textcolor[rgb]{0.40,0.40,0.40}{{#1}}}
    \newcommand{\BuiltInTok}[1]{{#1}}
    \newcommand{\ExtensionTok}[1]{{#1}}
    \newcommand{\PreprocessorTok}[1]{\textcolor[rgb]{0.74,0.48,0.00}{{#1}}}
    \newcommand{\AttributeTok}[1]{\textcolor[rgb]{0.49,0.56,0.16}{{#1}}}
    \newcommand{\InformationTok}[1]{\textcolor[rgb]{0.38,0.63,0.69}{\textbf{\textit{{#1}}}}}
    \newcommand{\WarningTok}[1]{\textcolor[rgb]{0.38,0.63,0.69}{\textbf{\textit{{#1}}}}}
    
    
    % Define a nice break command that doesn't care if a line doesn't already
    % exist.
    \def\br{\hspace*{\fill} \\* }
    % Math Jax compatability definitions
    \def\gt{>}
    \def\lt{<}
    % Document parameters
    \title{linear\_algebra\_review}
    
    
    

    % Pygments definitions
    
\makeatletter
\def\PY@reset{\let\PY@it=\relax \let\PY@bf=\relax%
    \let\PY@ul=\relax \let\PY@tc=\relax%
    \let\PY@bc=\relax \let\PY@ff=\relax}
\def\PY@tok#1{\csname PY@tok@#1\endcsname}
\def\PY@toks#1+{\ifx\relax#1\empty\else%
    \PY@tok{#1}\expandafter\PY@toks\fi}
\def\PY@do#1{\PY@bc{\PY@tc{\PY@ul{%
    \PY@it{\PY@bf{\PY@ff{#1}}}}}}}
\def\PY#1#2{\PY@reset\PY@toks#1+\relax+\PY@do{#2}}

\expandafter\def\csname PY@tok@gd\endcsname{\def\PY@tc##1{\textcolor[rgb]{0.63,0.00,0.00}{##1}}}
\expandafter\def\csname PY@tok@gu\endcsname{\let\PY@bf=\textbf\def\PY@tc##1{\textcolor[rgb]{0.50,0.00,0.50}{##1}}}
\expandafter\def\csname PY@tok@gt\endcsname{\def\PY@tc##1{\textcolor[rgb]{0.00,0.27,0.87}{##1}}}
\expandafter\def\csname PY@tok@gs\endcsname{\let\PY@bf=\textbf}
\expandafter\def\csname PY@tok@gr\endcsname{\def\PY@tc##1{\textcolor[rgb]{1.00,0.00,0.00}{##1}}}
\expandafter\def\csname PY@tok@cm\endcsname{\let\PY@it=\textit\def\PY@tc##1{\textcolor[rgb]{0.25,0.50,0.50}{##1}}}
\expandafter\def\csname PY@tok@vg\endcsname{\def\PY@tc##1{\textcolor[rgb]{0.10,0.09,0.49}{##1}}}
\expandafter\def\csname PY@tok@vi\endcsname{\def\PY@tc##1{\textcolor[rgb]{0.10,0.09,0.49}{##1}}}
\expandafter\def\csname PY@tok@mh\endcsname{\def\PY@tc##1{\textcolor[rgb]{0.40,0.40,0.40}{##1}}}
\expandafter\def\csname PY@tok@cs\endcsname{\let\PY@it=\textit\def\PY@tc##1{\textcolor[rgb]{0.25,0.50,0.50}{##1}}}
\expandafter\def\csname PY@tok@ge\endcsname{\let\PY@it=\textit}
\expandafter\def\csname PY@tok@vc\endcsname{\def\PY@tc##1{\textcolor[rgb]{0.10,0.09,0.49}{##1}}}
\expandafter\def\csname PY@tok@il\endcsname{\def\PY@tc##1{\textcolor[rgb]{0.40,0.40,0.40}{##1}}}
\expandafter\def\csname PY@tok@go\endcsname{\def\PY@tc##1{\textcolor[rgb]{0.53,0.53,0.53}{##1}}}
\expandafter\def\csname PY@tok@cp\endcsname{\def\PY@tc##1{\textcolor[rgb]{0.74,0.48,0.00}{##1}}}
\expandafter\def\csname PY@tok@gi\endcsname{\def\PY@tc##1{\textcolor[rgb]{0.00,0.63,0.00}{##1}}}
\expandafter\def\csname PY@tok@gh\endcsname{\let\PY@bf=\textbf\def\PY@tc##1{\textcolor[rgb]{0.00,0.00,0.50}{##1}}}
\expandafter\def\csname PY@tok@ni\endcsname{\let\PY@bf=\textbf\def\PY@tc##1{\textcolor[rgb]{0.60,0.60,0.60}{##1}}}
\expandafter\def\csname PY@tok@nl\endcsname{\def\PY@tc##1{\textcolor[rgb]{0.63,0.63,0.00}{##1}}}
\expandafter\def\csname PY@tok@nn\endcsname{\let\PY@bf=\textbf\def\PY@tc##1{\textcolor[rgb]{0.00,0.00,1.00}{##1}}}
\expandafter\def\csname PY@tok@no\endcsname{\def\PY@tc##1{\textcolor[rgb]{0.53,0.00,0.00}{##1}}}
\expandafter\def\csname PY@tok@na\endcsname{\def\PY@tc##1{\textcolor[rgb]{0.49,0.56,0.16}{##1}}}
\expandafter\def\csname PY@tok@nb\endcsname{\def\PY@tc##1{\textcolor[rgb]{0.00,0.50,0.00}{##1}}}
\expandafter\def\csname PY@tok@nc\endcsname{\let\PY@bf=\textbf\def\PY@tc##1{\textcolor[rgb]{0.00,0.00,1.00}{##1}}}
\expandafter\def\csname PY@tok@nd\endcsname{\def\PY@tc##1{\textcolor[rgb]{0.67,0.13,1.00}{##1}}}
\expandafter\def\csname PY@tok@ne\endcsname{\let\PY@bf=\textbf\def\PY@tc##1{\textcolor[rgb]{0.82,0.25,0.23}{##1}}}
\expandafter\def\csname PY@tok@nf\endcsname{\def\PY@tc##1{\textcolor[rgb]{0.00,0.00,1.00}{##1}}}
\expandafter\def\csname PY@tok@si\endcsname{\let\PY@bf=\textbf\def\PY@tc##1{\textcolor[rgb]{0.73,0.40,0.53}{##1}}}
\expandafter\def\csname PY@tok@s2\endcsname{\def\PY@tc##1{\textcolor[rgb]{0.73,0.13,0.13}{##1}}}
\expandafter\def\csname PY@tok@nt\endcsname{\let\PY@bf=\textbf\def\PY@tc##1{\textcolor[rgb]{0.00,0.50,0.00}{##1}}}
\expandafter\def\csname PY@tok@nv\endcsname{\def\PY@tc##1{\textcolor[rgb]{0.10,0.09,0.49}{##1}}}
\expandafter\def\csname PY@tok@s1\endcsname{\def\PY@tc##1{\textcolor[rgb]{0.73,0.13,0.13}{##1}}}
\expandafter\def\csname PY@tok@ch\endcsname{\let\PY@it=\textit\def\PY@tc##1{\textcolor[rgb]{0.25,0.50,0.50}{##1}}}
\expandafter\def\csname PY@tok@m\endcsname{\def\PY@tc##1{\textcolor[rgb]{0.40,0.40,0.40}{##1}}}
\expandafter\def\csname PY@tok@gp\endcsname{\let\PY@bf=\textbf\def\PY@tc##1{\textcolor[rgb]{0.00,0.00,0.50}{##1}}}
\expandafter\def\csname PY@tok@sh\endcsname{\def\PY@tc##1{\textcolor[rgb]{0.73,0.13,0.13}{##1}}}
\expandafter\def\csname PY@tok@ow\endcsname{\let\PY@bf=\textbf\def\PY@tc##1{\textcolor[rgb]{0.67,0.13,1.00}{##1}}}
\expandafter\def\csname PY@tok@sx\endcsname{\def\PY@tc##1{\textcolor[rgb]{0.00,0.50,0.00}{##1}}}
\expandafter\def\csname PY@tok@bp\endcsname{\def\PY@tc##1{\textcolor[rgb]{0.00,0.50,0.00}{##1}}}
\expandafter\def\csname PY@tok@c1\endcsname{\let\PY@it=\textit\def\PY@tc##1{\textcolor[rgb]{0.25,0.50,0.50}{##1}}}
\expandafter\def\csname PY@tok@o\endcsname{\def\PY@tc##1{\textcolor[rgb]{0.40,0.40,0.40}{##1}}}
\expandafter\def\csname PY@tok@kc\endcsname{\let\PY@bf=\textbf\def\PY@tc##1{\textcolor[rgb]{0.00,0.50,0.00}{##1}}}
\expandafter\def\csname PY@tok@c\endcsname{\let\PY@it=\textit\def\PY@tc##1{\textcolor[rgb]{0.25,0.50,0.50}{##1}}}
\expandafter\def\csname PY@tok@mf\endcsname{\def\PY@tc##1{\textcolor[rgb]{0.40,0.40,0.40}{##1}}}
\expandafter\def\csname PY@tok@err\endcsname{\def\PY@bc##1{\setlength{\fboxsep}{0pt}\fcolorbox[rgb]{1.00,0.00,0.00}{1,1,1}{\strut ##1}}}
\expandafter\def\csname PY@tok@mb\endcsname{\def\PY@tc##1{\textcolor[rgb]{0.40,0.40,0.40}{##1}}}
\expandafter\def\csname PY@tok@ss\endcsname{\def\PY@tc##1{\textcolor[rgb]{0.10,0.09,0.49}{##1}}}
\expandafter\def\csname PY@tok@sr\endcsname{\def\PY@tc##1{\textcolor[rgb]{0.73,0.40,0.53}{##1}}}
\expandafter\def\csname PY@tok@mo\endcsname{\def\PY@tc##1{\textcolor[rgb]{0.40,0.40,0.40}{##1}}}
\expandafter\def\csname PY@tok@kd\endcsname{\let\PY@bf=\textbf\def\PY@tc##1{\textcolor[rgb]{0.00,0.50,0.00}{##1}}}
\expandafter\def\csname PY@tok@mi\endcsname{\def\PY@tc##1{\textcolor[rgb]{0.40,0.40,0.40}{##1}}}
\expandafter\def\csname PY@tok@kn\endcsname{\let\PY@bf=\textbf\def\PY@tc##1{\textcolor[rgb]{0.00,0.50,0.00}{##1}}}
\expandafter\def\csname PY@tok@cpf\endcsname{\let\PY@it=\textit\def\PY@tc##1{\textcolor[rgb]{0.25,0.50,0.50}{##1}}}
\expandafter\def\csname PY@tok@kr\endcsname{\let\PY@bf=\textbf\def\PY@tc##1{\textcolor[rgb]{0.00,0.50,0.00}{##1}}}
\expandafter\def\csname PY@tok@s\endcsname{\def\PY@tc##1{\textcolor[rgb]{0.73,0.13,0.13}{##1}}}
\expandafter\def\csname PY@tok@kp\endcsname{\def\PY@tc##1{\textcolor[rgb]{0.00,0.50,0.00}{##1}}}
\expandafter\def\csname PY@tok@w\endcsname{\def\PY@tc##1{\textcolor[rgb]{0.73,0.73,0.73}{##1}}}
\expandafter\def\csname PY@tok@kt\endcsname{\def\PY@tc##1{\textcolor[rgb]{0.69,0.00,0.25}{##1}}}
\expandafter\def\csname PY@tok@sc\endcsname{\def\PY@tc##1{\textcolor[rgb]{0.73,0.13,0.13}{##1}}}
\expandafter\def\csname PY@tok@sb\endcsname{\def\PY@tc##1{\textcolor[rgb]{0.73,0.13,0.13}{##1}}}
\expandafter\def\csname PY@tok@k\endcsname{\let\PY@bf=\textbf\def\PY@tc##1{\textcolor[rgb]{0.00,0.50,0.00}{##1}}}
\expandafter\def\csname PY@tok@se\endcsname{\let\PY@bf=\textbf\def\PY@tc##1{\textcolor[rgb]{0.73,0.40,0.13}{##1}}}
\expandafter\def\csname PY@tok@sd\endcsname{\let\PY@it=\textit\def\PY@tc##1{\textcolor[rgb]{0.73,0.13,0.13}{##1}}}

\def\PYZbs{\char`\\}
\def\PYZus{\char`\_}
\def\PYZob{\char`\{}
\def\PYZcb{\char`\}}
\def\PYZca{\char`\^}
\def\PYZam{\char`\&}
\def\PYZlt{\char`\<}
\def\PYZgt{\char`\>}
\def\PYZsh{\char`\#}
\def\PYZpc{\char`\%}
\def\PYZdl{\char`\$}
\def\PYZhy{\char`\-}
\def\PYZsq{\char`\'}
\def\PYZdq{\char`\"}
\def\PYZti{\char`\~}
% for compatibility with earlier versions
\def\PYZat{@}
\def\PYZlb{[}
\def\PYZrb{]}
\makeatother


    % Exact colors from NB
    \definecolor{incolor}{rgb}{0.0, 0.0, 0.5}
    \definecolor{outcolor}{rgb}{0.545, 0.0, 0.0}



    
    % Prevent overflowing lines due to hard-to-break entities
    \sloppy 
    % Setup hyperref package
    \hypersetup{
      breaklinks=true,  % so long urls are correctly broken across lines
      colorlinks=true,
      urlcolor=blue,
      linkcolor=darkorange,
      citecolor=darkgreen,
      }
    % Slightly bigger margins than the latex defaults
    
    \geometry{verbose,tmargin=1in,bmargin=1in,lmargin=1in,rmargin=1in}
    
    

    \begin{document}
    
    
    \maketitle
    
    

    
    \section{Linear Algebra: Full Review}\label{linear-algebra-full-review}

This review will be composed initially of very basic and simple examples
of notation and matrix/vector arithmetic. Where appropriate, written
descriptions and python graphics/calculations will be included

    \subsection{Chapter 1: Terms and
Definitions}\label{chapter-1-terms-and-definitions}

\subsection{1.1 Column Vectors}\label{column-vectors}

They're exactly what they sound like :) Column vectors are vectors
written vertically.

ex.

\[ c\vec{v} + d\vec{w} = c\begin{bmatrix} 1 \\ 1 \end{bmatrix} + d\begin{bmatrix} 2 \\ 3 \end{bmatrix} = \begin{bmatrix} c + 2d \\ c + 3d \end{bmatrix}\]

\[ \Rightarrow 2\vec{v} + \vec{w} = 2\begin{bmatrix} 1 \\ 1 \end{bmatrix} + \begin{bmatrix} 2 \\ 3 \end{bmatrix} = \begin{bmatrix} 4 \\ 5 \end{bmatrix}\]

    \emph{``Vectors correspond to points \textbf{but are independent of
coordinate systems}''}

*images and quote taken from Dr.~Carl Gardner's lecture notes

    \subsection{1.2 Lengths and dot
products}\label{lengths-and-dot-products}

\[||\vec{v}|| = \sqrt{\langle \vec{v} \, , \vec{v} \rangle}\]

In the above equation, note that the left-hand expression is the
euclidean norm (sometimes referred to as the ``magnitude'') of the
vector \(\vec{v},\) and the right hand expression (under the radical) is
the inner product (also called the dot product).

\[\langle \vec{v} \, , \vec{w} \rangle = v^{\, T}w = \begin{bmatrix} v_{1} & v_{2} & v_{3} \end{bmatrix} \begin{bmatrix} w_{1} \\ w_{2} \\ w_{3} \end{bmatrix} = v_{1}w_{1} + v_{2}w_{2} + v_{3}w_{3}\]

\[=||\vec{v}|| \, ||\vec{w}|| \, cos \, \theta\]

Dot products are \emph{commutative}:

\[\langle \vec{v} \, , \vec{w} \rangle = \langle \vec{w} \, , \vec{v} \rangle = \vec{v} \cdot \vec{w} = \vec{w} \cdot \vec{v}\]

Unit vectors take the form:

\[\hat{i} = \begin{bmatrix} 1 \\ 0 \\ 0 \end{bmatrix} \hspace{.2cm} \hat{j} = \begin{bmatrix} 0 \\ 1 \\ 0 \end{bmatrix} \hspace{.2cm} \hat{k} = \begin{bmatrix} 0 \\ 0 \\ 1 \end{bmatrix} \]

    \textbf{Examples:}

To find the unit vector of a given vector \(\vec{u},\) where:

\[\vec{u} = \begin{bmatrix} 2 & 2 & 1 \end{bmatrix} \]

first solve for the norm of \(\vec{u}:\)

\[\Rightarrow || \, \vec{u} \, || = \sqrt{2^2 + 2^2 +1^2} = \sqrt{9} = 3\]

then divide out the norm from each element of \(\vec{u}:\)

\[ \Leftarrow\Rightarrow \hat{u} = \dfrac{\vec{u}}{||\, \vec{u} \,||} = \begin{bmatrix} \dfrac{2}{3} & \dfrac{2}{3} & \dfrac{1}{3} \end{bmatrix}\]

\textbf{\emph{Random fact 1.1}} The unit circle vector is written:

\[\vec{u}_c = \begin{bmatrix} cos \, \theta \\ sin \, \theta \end{bmatrix}\]

    \textbf{Cauchy-Schwarz Inequality}

\[ \vec{v} \cdot \vec{w} \leq || \vec{v} || \cdot || \vec{w} ||\]

\textbf{Triangle Inequality}

\[ || \vec{v} + \vec{w} || \leq ||\vec{v}|| + ||\vec{w}||\]

    \subsection{1.3 Matrices}\label{matrices}

    \textbf{Matrix Properties}

\[ I = \begin{bmatrix} 1 & 0 \\ 0 & 1 \end{bmatrix}\]

If A has an inverse, then: \[ AA^{-1} = I = A^{-1}A\]

and

\[ A\vec{x} = \vec{b} \]\\
\[ \Rightarrow \vec{x} = A^{-1} \vec{b} \]

A system of equations is said to be underdetermined if there are fewer
equations than there are dimensions to the system. In matrix notation:

\[\left[\begin{array}{cc|c} 1 & -2 & 1 \\ 0 & 0 & 0 \end{array}\right]\]

Likewise, a system of equations is referred to as overdetermined if it
has more equations than variables:

\[\left[\begin{array}{cc|c} 1 & -2 & 0 \\ 3 & 2 & 0 \\ 2 & 1 & 0 \end{array}\right]\]

    \textbf{\emph{Examples of linear equations}}

\textbf{Difference Matrix:}

The following matrix D is referred to as a difference matrix:

\[D = \begin{bmatrix}  1 & 0 & 0 \\ -1 & 1 & 0 \\ 0 & -1 & 1 \end{bmatrix}\]

\[D\vec{x} = \begin{bmatrix}  1 & 0 & 0 \\ -1 & 1 & 0 \\ 0 & -1 & 1 \end{bmatrix} \begin{bmatrix}  x_1 \\ x_2 \\ x_3 \end{bmatrix} = \begin{bmatrix}  b_1 \\ b_2 \\ b_3 \end{bmatrix}\]

\[ = \begin{bmatrix}  x_1 + (0) \, x_2 + (0) \, x_3 \\  -x_1 + x_2 + (0) \, x_3 \\ (0) \, x_1 - x_2 + x_3 \end{bmatrix} \]

\[ = \begin{bmatrix}  x_1 \\ x_2 - x_1 \\ x_3 - x_2 \end{bmatrix}\]

Therefore,

\[ x_1 = b_1 \] \[ x_2 - x_1 = b_2\] \[ x_3 - x_2 = b_3\]

\[\Rightarrow x_1 = b_1 \] \[\Rightarrow x_2 = b_1 + b_2\]
\[\Rightarrow x_3 = b_1 + b_2 +b_3\]

\[\Rightarrow \vec{x} = S \vec{b} = \begin{bmatrix} 1 & 0 & 0 \\ 1 & 1 & 0 \\ 1 & 1 & 1 \end{bmatrix} \, \vec{b}, \]
where S is the sum matrix, as well as the inverse matrix of D.

\textbf{Check:}

\[DS = DD^{-1} = \begin{bmatrix}  1 & 0 & 0 \\ -1 & 1 & 0 \\ 0 & -1 & 1 \end{bmatrix} \begin{bmatrix} 1 & 0 & 0 \\ 1 & 1 & 0 \\ 1 & 1 & 1 \end{bmatrix} = \begin{bmatrix}  1 & 0 & 0 \\ 0 & 1 & 0 \\ 0 & 0 & 1 \end{bmatrix} \]

\textbf{Cycle Matrix:}

\[C \, \vec{x} = \begin{bmatrix}  1 & 0 & -1 \\ -1 & 1 & 0 \\ 0 & -1 & 1 \end{bmatrix} \begin{bmatrix}  x_1 \\ x_2 \\ x_3 \end{bmatrix} = \begin{bmatrix}  x_1 - x_3 \\ x_2 - x_1 \\ x_3 - x_2 \end{bmatrix} = \vec{b} \]

For this matrix, there are either infinitely many solutions for
\(\vec{x},\) or no solutions.

One can see that this system is underdetermined (i.e.~the number of
unique equations describing the system is less than the dimensionality
of the system) by pivoting the matrix a number of times to get:

\[C = \begin{bmatrix}  1 & 0 & -1 \\ -1 & 1 & 0 \\ 0 & -1 & 1 \end{bmatrix} = \begin{bmatrix}  1 & 0 & -1 \\ 0 & 1 & -1 \\ 0 & 0 & 0 \end{bmatrix} \]

\textbf{Independence and Singularity}

The \emph{\(n \times n\)} (``square'') matrix A -- where, in the
2-dimensional case: \[A = \begin{bmatrix} a & b \\ c & d \end{bmatrix}\]

is said to be invertible if \(det \{A\} = ad - bc \neq 0,\) the columns
of A are independent, and A has \emph{n} pivots. A is referred to as
singular if all of the above qualities are negated (i.e.
\(det\{A\} = 0,\) the columns of A are dependent, and A has less than
\emph{n} pivots). We'll return to this idea of independence again after
we've wet our beaks with some matrix arithmetic. If A is indeed found to
be invertible, \(\, A^{-1}\) exists and is unique, \$ , AA\^{}\{-1\} = I
= A\^{}\{-1\}A,\$ and the unique solution to \$ A\vec{x} = \vec{b}\$ is
\(\vec{x} = A^{-1} \vec{b}\)

\emph{Note: pivots may seem a little strange (especially if you read
from the wikipedia entry). However, the concept is relatively
straightforward. A pivot is simply the first (as in left-most) non-zero
entry in a row of a matrix. Voila!}

    \#\# Chapter 2: Solving Linear Equations

\#\# 2.1 Vectors and Linear Equations

Consider the following system of equations:

\[x - 2y = 1\] \[3x + 2y = 11\]

This corresponds to the augmented matrix:

\[A = \left[\begin{array}{cc|c} 1 & -2 & 1 \\ 3 & 2 & 11 \end{array}\right] \]

    This matrix can be put into \emph{Reduced Row Echelon Form (RREF)} by
way of row manipulations such that it's solution can be more readily
gleaned:

\[A = \left[\begin{array}{cc|c} 1 & -2 & 1 \\ 3 & 2 & 11 \end{array}\right] \]

\[ = \left[\begin{array}{cc|c} 1 & -2 & 1 \\ 0 & 8 & 8 \end{array}\right]\]

\[ = \left[\begin{array}{cc|c} 1 & -2 & 1 \\ 0 & 1 & 1 \end{array}\right]\]

\[ = \left[\begin{array}{cc|c} 1 & 0 & 3 \\ 0 & 1 & 1 \end{array}\right]\]

Note that, for each variable in the system, there is a corresponding
pivot equal to one. Additionally, notice that these pivots are aligned
along the diagonal passing from the top left to the bottom right-most
entries of the coefficient matrix. The resultant augmented matrix can be
interpreted in one of two ways. By evaluating the rows, we see that the
system is two lines that intersect at \((3,1);\) by evaluating the
columns, we see a linear combination of two vectors set equal to some
solution vector b:

\[ x\begin{bmatrix} 1 \\ 3 \end{bmatrix} + y\begin{bmatrix} -2 \\ 2 \end{bmatrix} = \begin{bmatrix} 1 \\ 11 \end{bmatrix} \]

\[ \Rightarrow x\begin{bmatrix} 1 \\ 0 \end{bmatrix} + y\begin{bmatrix} -2 \\ 8 \end{bmatrix} = \begin{bmatrix} 1 \\ 8 \end{bmatrix} \]

\[ \Rightarrow x\begin{bmatrix} 1 \\ 0 \end{bmatrix} + y\begin{bmatrix} -2 \\ 1 \end{bmatrix} = \begin{bmatrix} 1 \\ 1 \end{bmatrix}\]

\[\Rightarrow x\begin{bmatrix} 1 \\ 0 \end{bmatrix} + y\begin{bmatrix} 0 \\ 1 \end{bmatrix} = \begin{bmatrix} 3 \\ 1 \end{bmatrix}\]

    Now, at this point, you may be asking, ``Taylor, what the hell do you
mean by `row manipulations?'\,'' Excellent question, dear reader. Matrix
arithmetic entails a certain set of rules that govern the ways you can
move and rescale, add and subtract rows in a matrix without altering its
solution: 1. Any row can be multiplied by some scalar value \(\alpha\)
such that all entries in that row are rescaled by \(\alpha\) 2. Any row
can be added or subtracted from any scalar multiple of any other row 3.
Any two rows can swap places in the matrix

Applying these rules to the above example, we see that, in my
calculations, I:

\begin{enumerate}
\def\labelenumi{\arabic{enumi}.}
\tightlist
\item
  Subtracted three times row one from row two \[3R_1 - R_2\]
\item
  Divided row two by the scalar value eight\\
  \[\dfrac{R_2}{8}\]
\item
  Added two times row two to row one \[2R_2 + R_1\]
\end{enumerate}

and presto! We have the RREF of the matrix.


    % Add a bibliography block to the postdoc
    
    
    
    \end{document}
